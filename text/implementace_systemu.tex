\chapter{Implementácia systému}

Z kapitoly \ref{vyber} vyšlo že Linode je najlepšia voľba pre virtuálne prostredie na nasadenie mojho malého e-shopu. Vytvorenie prostredia bolo jednoduché ako je možné vidieť na obrázkoch \ref{img:one} a \ref{img:two}.

\begin{figure}[ht!]
  \includegraphics[width=\linewidth]{images/1.png}
  \caption{Prvá časť vytvorenia virtuálneho prostredia}
  \label{img:one}
\end{figure}

\begin{figure}[ht!]
  \includegraphics[width=\linewidth]{images/2.png}
  \caption{Druhá časť vytvorenia virtuálneho prostredia}
  \label{img:two}
\end{figure}

\noindent Užívateľské rozhranie ovládacieho panelu pre Linode je veľmi intuitívne a prehľadné ako je zase možné vidieť na obrázkoch \ref{img:three} a \ref{img:four}.

\begin{figure}[ht!]
  \includegraphics[width=\linewidth]{images/3.png}
  \caption{Prehľad sieťového provozu}
  \label{img:three}
\end{figure}

\begin{figure}[ht!]
  \includegraphics[width=\linewidth]{images/4.png}
  \caption{Prehľad systému}
  \label{img:four}
\end{figure}

Ako operačný systém zvolím Linux Debian keďže mám skúsenosti s týmto operačným systémom a je na ňom jednoduchšie vytvárať a hostovať webové aplikácie.

\section{Technické kroky implementácie systému}

Ako prvé by som nastavil potrebné bezpečnostné opatrenia, konkrétne by som nastavil firewall pomocou programu \texttt{iptables} ktorý umožňuje veľmi detailné omedzenie sieťovej prevádzky. Napríklad zakážem všetky porty okrem 22 (ssh) a 443 (https), zakážem ICMP protokol a IPv6. Ďalej nastavím prihlásenie iba cez ssh klúč ktorý si dôkladne uložím. Nainštalujem všetky potrebné nástroje a aplikácie na spustenie mojej vyvinutej aplikácie. V tomto prípade by to boli programy \texttt{npm}, \texttt{nvm}, \texttt{nginx}. Taktiež by som zapol automatický update balíčkov z verziami ktoré obsahujú bezpečnostné vylepšenia aby bol systém up to date. Prístup do vnútornej časti systému ktorý bude ukladať objednávky bude možný len pomocou hesla a dvoj faktorovej autentizácie.

Ďalší krok bude zakúpenie doménového mena pomocou stránky \cite{domain}. Aby bolo možné doménu využívať doménu na internete, bude potreba vytvoriť globálny preklad domény na IP adresu môjho serveru, na toto využijem Google Cloud DNS \cite{dns}. Keďže bude stránka podporovať protokol HTTPS, potrebuje aj X509 certifikát. Certifikát budem kupovať na moje doménové meno cez ssls.com \cite{cert}. Predpokladaná suma pre doménu, DNS preklad a certifikát je 9\,\texteuro~na mesiac. V cene pre DNS preklad je 10 miliónov prekladov za mesiac čo by na začiatku predaja mojich služieb bude stačiť. Platobná braná ktorú ingergrujem do mojej stránky bude paypal. Pre frontend aplikácie použijem tému Now UI Kit \cite{vue_kit}.

Linode a Vue framework som si vyskúšal spustiť naživo a momentálne je len základná stránka dostupná na adrese \url{http://172.105.69.134:8080}.