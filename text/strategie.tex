\chapter{Stratégia}

Činnosť mojej živnosti nebude produkt ale služba. Cieľom služby bude priniesť do informačného systému ľahkú rozšíritelnosť, dostupnosť služieb a jednoduché pozorovanie stavu systému. Jadro ponúkanej služby bude vytvorenie produkčného prostredia pre akýchkoľvek informačný systém, e-shop atď. pomocou systému Kubernetes. Kubernetes je orchestračný nástroj ktorý umožnuje nasadenie aplikácie pomocou docker kontainerov ktoré obsahujú komopnenty aplikácie. 

Bežný e-shop potrebuje k funkčnosti databázu, samotnú webovú aplikáciu a ak je e-shop večší, tak potrebuje aj kašovací server a load balancing. Všetky tieto komopnenty musia spolu komunikovať aby bol e-shop funkčný. Ak majitel alebo programátor e-shopu niesú zoznámený z technológiamy ako Docker a Kubernetes, budú mať všetky komonenty spustené na jednom serveri bez virtualizácie. Tento spôsob nasadenia je tažko udržovaťelný. Súčastou mojej služby bude prekonvertovanie každého komponentu do samostatnej jednotky (Docker image) a pomocou Kubernetes systému vytvorím produkčné prostredie pre daný eshop.

Čím viac bude informačný systém rásť, bude ho exponenciálne tažšie udržovať a ďalej ho rozšírovať. Kubernetes dovoľuje dynamicky zvišovať dostupný výpočetný výkon, dokáže automaticky spustiť novú verziu informačného systému bez žiadneho výpadku, zvyšovať inštancie komponentov informačného systému a tým rozdeliť zátaž na viacej jednodlivých inštancí rovnakého komponentu alebo automaticky reštartovať akýkoľvek komponent pri vápadku. 

Zákazník si bude môcť taktiež vybrať možnosť dlhodobej podpory k môjmu vytvorenému systému Kubernetes. Ak by sa vyskitol problém, budem dostupný a schopný ho opraviť.Budem si udržovať databázu vytvorených Kubernetes systémov a riešenie problém sa mi jednoznačne ulachčí. Nebudem ale ponúkať podboru Kubernetes systémom ktoré vytvoril niekto iný keďže pravdepodobne nebudem dostupnú dokumentáciu k spôsobou nasadenia.

Ďalšiu službu ktorú budem ponúkať bude vytvorenie webového portálu pomocou ktorého bude možné sledovať dôležité informácie o chode systému ako napríklad vytaženie systému, počet návštev, uptime atď.


\section{Doručovanie}

Nebudem doručovať ziadny fyzický produkt.

\section{Cielený zákazníci}

Cielený zákazníci budú jednotlivci, skupiny ludí alebo organizácie ktoré už majú ustanovený informačný systém a stáli výskyt navštevníkov alebo zákazníkov. Moju službu budú výhladávať pretože už pradepodobne nebudú schopný ďalej rozšírovať jejich informačný systém alebo ich to stojí vela času a penazí. Ďalej budem cieliť svoje služby aj na zákazníkov ktorý si práve vytvárajú čerstvý informačný systém a potrebujú k nemu vytvoriť spolahlivé produkčné prostredie.

Služby budú dostupné zákazníkom iba celej európy pokial bude možné zo zákazníkom komunikovať anglicky.

\section{Infraštruktúra}

Ponúkanú službe je možné rozdeliť na dva typy z pohľadu infraštruktúry:
\begin{itemize}
  \item \textbf{Infraštruktúra neviditelná zákazníkovy}\,--\,Hardware na ktorých bude hostovaný systém Kubernetes bude môj vyrtuálny stroj ktorý budem mať prenajatý.
  \item \textbf{Zákazníkova infraštruktúra}\,--\,Ak už má zákazník dostatočné výkonný hardware na ktorom chce hostovať svoj informačný systém, využijem ho namiesto mojich prenajatých serverov.
\end{itemize}

\section{Cenová stratégia}



% \begin{table}[htbp]
%   \centering
%   \caption{Rozdelenie predajnej ceny}
%   \label{rozdelenie_ceny}
%   \begin{tabular}{|c|c|}
%     \hline
%     Percentá [\%] & Vysvetlenie \\
%     \hline
%     \hline
%     100 & Nákup dielu \\
%     \hline
%     15 & Poštovné (Môže sa líšiť) \\
%     \hline
%     15 & Zisk (Môže sa líšiť) \\
%     \hline
%   \end{tabular}
% \end{table}

