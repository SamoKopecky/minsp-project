\chapter{Stratégia}

V nasledujúcich sekciách je popísaná stratégia môjho živnostníctva.

\section{Definícia hlavnej služby}
Činnosť mojej živnosti nebude produkt ale služba. Cieľom služby bude priniesť do informačného systému ľahkú rozšíriteľnosť, dostupnosť služieb a jednoduché pozorovanie stavu systému. Jadro ponúkanej služby bude vytvorenie produkčného prostredia pre akýchkoľvek informačný systém, e-shop atď. pomocou systému Kubernetes. Kubernetes je orchestračný nástroj ktorý umožňuje nasadenie aplikácie pomocou docker kontajnerov ktoré obsahujú komponenty aplikácie.

Bežný e-shop potrebuje k funkčnosti databázu, samotnú webovú aplikáciu a ak je e-shop väčší, tak potrebuje aj kašovací server a load balancing. Všetky tieto komponenty musia spolu komunikovať aby bol e-shop funkčný. Ak majiteľ alebo programátor e-shopu nie sú zoznámený z technológiami ako Docker a Kubernetes, budú mať všetky komponenty spustené na jednom serveri bez virtuálizácie. Tento spôsob nasadenia je ťažko udržovatelný. Súčasťou mojej služby bude konvertovanie každého komponentu do samostatnej jednotky (Docker image) a pomocou Kubernetes systému vytvorím produkčné prostredie pre daný eshop.

Čím viac bude informačný systém rásť, bude ho exponenciálne ťažšie udržovať a ďalej ho rozširovať. Kubernetes dovoľuje dynamicky zvyšovať dostupný výpočetný výkon, dokáže automaticky spustiť novú verziu informačného systému bez žiadneho výpadku, zvyšovať inštancie komponentov informačného systému a tým rozdeliť zátaž na viacej jednotlivých inštancií rovnakého komponentu alebo automaticky reštartovať akýkoľvek komponent pri výpadku. 

\section{Ostatné služby}

Zákazník si bude môcť taktiež vybrať možnosť dlhodobej podpory k môjmu vytvorenému systému Kubernetes. Ak by sa vyskytol problém, budem dostupný a schopný ho opraviť. Budem si udržovať databázu vytvorených Kubernetes systémov a riešenie problém sa mi jednoznačne usnadní. Nebudem ale ponúkať podporu Kubernetes systémom ktoré vytvoril niekto iný keďže pravdepodobne nebudem dostupnú dokumentáciu k spôsobu nasadenia.

Ďalšiu doplnkovú službu ktorú budem ponúkať bude vytvorenie webového portálu pomocou ktorého bude možné sledovať dôležité informácie o chode systému ako napríklad vyťaženie systému, počet návštev, uptime atď.

Službu budem poskytovať menšiemu počtu zákazníkov keďže už jedna objednávka môže byť časovo náročná, hlavne ak budem ďalej ponúkať aj dlhodobú podporu systému.

\section{Doručovanie}

Nebudem doručovať žiadny fyzický produkt.

\section{Cielený zákazníci}

Cielený zákazníci budú jednotlivci, skupiny ľudí alebo organizácie ktoré už majú ustanovený informačný systém a stáli výskyt návštevníkov alebo zákazníkov. Moju službu budú vyhľadávať pretože už pravdepodobne nebudú schopný ďalej rozširovať ich informačný systém alebo ich to stojí veľa času a peňazí. Ďalej budem cieliť svoje služby aj na zákazníkov ktorý si práve vytvárajú čerstvý informačný systém a potrebujú k nemu vytvoriť spoľahlivé produkčné prostredie.

Služby budú dostupné zákazníkom Európy pokiaľ bude možné zo zákazníkom komunikovať anglicky.

\section{Infraštruktúra}

Ponúkanú službu je možné rozdeliť na dva typy z pohľadu infraštruktúry:
\begin{itemize}
  \item \textbf{Infraštruktúra neviditeľná zákazníkovi}\,--\,Hardware na ktorých bude postavený systém Kubernetes bude môj virtuálny stroj ktorý budem mať prenajatý.
  \item \textbf{Zákazníkova infraštruktúra}\,--\,Ak už má zákazník dostatočné výkonný hardware na ktorom chce hostovať svoj informačný systém, využijem ho namiesto mojich prenajatých serverov.
\end{itemize}

\section{Cenová stratégia}
Cena mojej služby bude hlavne záležať na tom čo všetko bude zákazník od mňa očakávať a aký veľký informačný systém budem nasadzovať alebo udržovať. Ak bude zákazník iba žiadať nasadenie systému Kubernetes, cena služby bude väčší jednorázový poplatok v rozmezí 500 až 800\,\texteuro\ v pomere veľkosti informačného systému. Ak bude ale ďalej očakávať aj dlhodobú podporu, mesačný poplatok bude v rozsahu 100 až 150\,\texteuro. Ak bude súčasťou služby aj infraštruktúra, zákazník si bude platiť infraštruktúru samostatne.

